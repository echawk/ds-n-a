\documentclass{article}
\usepackage[utf8]{inputenc}
\usepackage{amsmath,amsfonts,amssymb,amsthm}
\usepackage{graphics}
\usepackage{tikz}
\usepackage{minted}
\usepackage[margin=1in]{geometry}

\setminted[]{
	xleftmargin=.75cm,
	xrightmargin=.75cm,
	frame=single,
	framesep=.25cm,
	linenos,
	tabsize=4,
	breaklines
}

\title{Homework \#1}
\author{CS-495: Data Structures and Algorithms}

\begin{document}
\date{}
\maketitle

Consider the following code snippets. Give the Big O, Big $\Theta$, and Big $\Omega$ \textbf{time complexity} for each snippet. Briefly explain what the code snippet is doing.
\begin{enumerate}
	% Big Omega: $n$
	% Big Theta: $n$
	% Big O    : 1
	\item %Sums numbers from 0 to n
	\begin{minted}{C}
		int func1(int n){
			int s = 0;

			for(int i = 0; i < n; i++)
				s += i;

			return s;
		}
	\end{minted}
	% Big Omega: $n^2$
	% Big Theta: $n$
	% Big O    : 1
	\item %Matrix addition
	\begin{minted}{C}
		void func2(int row, int col, int a[row][col], int b[row][col]){
			for(int i = 0; i < row; i++){
				for(int j = 0; j < col; j++){
					a[i][j] += b[i][j];
				}
			}
		}
	\end{minted}
	% Big Omega: $n^2$
	% Big Theta: $n^2$
	% Big O    : $n$
	\item %I don't remember what this is?
	\begin{minted}{C}
		void func3(int size, int a[size]){
			int key, j;	
			for(int i = 1; i < size; i++){
				key = a[i];
				j = i - 1;

				while(j >= 0 && a[j] > key){
					a[j + 1] = a[j];
					j = j - 1;
				}

				a[j + 1] = key;
			}
		}
	\end{minted}
	\item \textbf{Note:} Assume that func4 defines and is passed all arguments to make this code valid.
	\begin{minted}{C}
		void func4(...){
			func1(n);
			func2(row, col, a[row][col], b[row][col]);
			func3(size, a[size]);
		}
	\end{minted}
\end{enumerate}
Write an algorithm that satisfies the given Big O, Big $\Theta$, or Big $\Omega$ for the given complexity and explain how/why it meets the requirement.
\begin{enumerate}
	\item O(1) space complexity
	\item O($n^2$) time complexity
	\item $\Theta(n\log n)$ space complexity
	\item $\Omega(n)$ time complexity
\end{enumerate}

\end{document}