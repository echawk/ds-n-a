\documentclass[11pt]{article}
	\title{\textbf{Sorting Algorithm Notes}}
	\author{Christian Garcia}
	\date{}
	
	\addtolength{\topmargin}{-3cm}
    \addtolength{\textheight}{3cm}
\begin{document}

\maketitle
\thispagestyle{empty}

\section{Why are sorting algorithms important?}
Some useful explanations from \emph{Introduction to Algorithms} on how sorting algorithims are relevant and useful.\footnote{Thomas H. Cormen et al., Introduction to Algorithms (The MIT Press, 2022), 113.}

\begin{itemize}
\item{Inevitabily, data will need sorted for some reason or another. There are numerous sets of data that need sorted in practical applications. Some examples are: students at a university being sorted by graduation year, banks sorting checks by check number, etc.}
\item{Sorting algorithms are used as a key subroutine in many programs. Assume you are editing a screen with multiple layers of pixels. Using a sorting algorithm is vital to edit the pixels from the lowest layer before editing pixels from the highest layer.}
\item{Sorting algorithms employ unique solutions to sort data which can provide different use cases for each algorithm depending on the data. Some algorithms use a divide and conquer method whereas others may use simple comparisons and swaps to sort the data.}
\item{Sorting algorithms can be used for proving mathematical issues such as upper and lower bounds. With these bounds established, we can conclude that the sorting algorithms are asymptotically optimal.}
\item{Sorting algorithms can highlight many issues with the engineering of computing devices. Sorting algorithms are meant to be quick and efficient. The factors that are important to stress here are: keys and satellite data, memory hierarchy (i.e. caches and virtual memory), and the software environment.}
\end{itemize}


\section{Why use sorting algorithms in the first place?}
\addtolength{\itemindent}{0.80cm}
\itemsep0em 
\begin{itemize}
\item{Organize messy data into clean, easy to use data}
\item{Collect data points like lower/upper bounds, medians, etc.}
\item{Use sorted data to influence decisions based on ratios}
\end{itemize}


\section{Sorting algorithms we will cover, how they work, and when to use them.}
\begin{itemize}
\addtolength{\itemindent}{0.80cm}
\itemsep0em 

\item{Bucket}
\footnote{“Bucket Sort,” GeeksforGeeks (GeeksforGeeks, February 1, 2023), https://www.geeksforgeeks.org/bucket-sort-2/.}
\begin{itemize}
\addtolength{\itemindent}{0.80cm}
\itemsep0em 
\item{How does this algorithm work?}
\begin{itemize}
\addtolength{\itemindent}{0.80cm}
\itemsep0em 
\item{Create \emph{n} empty buckets (arrays)}
\item{Insert each piece of data into the appropriate bucket}
\item{Sort each individual bucket using another sorting algorithm (typically insertion sort)}
\item{Concatenate each bucket into one sorted array}
\end{itemize}
\item{When should you use this sorting algorithm?}
\begin{itemize}
\addtolength{\itemindent}{0.80cm}
\itemsep0em 
\item{Use when the data is distributed uniformly over a range (i. e. data from 0.0 to 1.0)}
\end{itemize}
\end{itemize}


\item{Cocktail}
\footnote{“Cocktail Sort,” GeeksforGeeks (GeeksforGeeks, July 19, 2022), https://www.geeksforgeeks.org/cocktail-sort/.}
\begin{itemize}
\addtolength{\itemindent}{0.80cm}
\itemsep0em 
\item{How does this algorithm work?}
\begin{itemize}
\addtolength{\itemindent}{0.80cm}
\itemsep0em 
\item{Loop through the array comparing to see if $ a[i] > a[i+1] $ }
\item{Loop through the array starting from the end of the array and comparing to see if $ a[i] > a[i+1] $}
\item{Repeat until the array is sorted}
\end{itemize}
\item{When should you use this sorting algorithm?}
\begin{itemize}
\addtolength{\itemindent}{0.80cm}
\itemsep0em 
\item{Useful for large arrays of data. Skips unncessary iterations unlike bubble sort.}
\end{itemize}
\end{itemize}

\item{Comb}
\footnote{“Comb Sort,” GeeksforGeeks (GeeksforGeeks, January 10, 2023), https://www.geeksforgeeks.org/comb-sort/.}
\begin{itemize}
\addtolength{\itemindent}{0.80cm}
\itemsep0em 
\item{How does this algorithm work?}
\begin{itemize}
\addtolength{\itemindent}{0.80cm}
\itemsep0em 
\item{Loop through the array comparing to see if $a[i] > a[i + gap] $}
\item{Typical gap is 1.3}
\end{itemize}
\item{When should you use this sorting algorithm?}
\begin{itemize}
\addtolength{\itemindent}{0.80cm}
\itemsep0em 
\item{On datasets that are not large. Remember, comb sort is only a slight improvment over bubble sort.}
\end{itemize}
\end{itemize}

\item{Counting}
\footnote{“Counting Sort,” GeeksforGeeks (GeeksforGeeks, February 2, 2023), https://www.geeksforgeeks.org/counting-sort/.}
\begin{itemize}
\addtolength{\itemindent}{0.80cm}
\itemsep0em 
\item{How does this algorithm work?}
\begin{itemize}
\addtolength{\itemindent}{0.80cm}
\itemsep0em 
\item{Create a separate array for counting the number of times a value occurs in the array of data}
\item{}
\end{itemize}
\item{When should you use this sorting algorithm?}
\begin{itemize}
\addtolength{\itemindent}{0.80cm}
\itemsep0em 
\item{}
\end{itemize}
\end{itemize}

\item{Counting}
\item{Gnome}
\item{Heap}
\item{Insertion}
\item{Merge}
\item{Quick}
\item{Radix}
\item{Shell}
\end{itemize}

\end{document}